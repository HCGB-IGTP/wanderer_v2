\documentclass{article}

\usepackage[a4paper, total={6in, 8in}]{geometry}
\usepackage{url}

\title{Wanderer: integration and visualization of TCGA Data}
\author{Anna D{\'i}ez-Villanueva$^{*}$, Izaskun Mallona$^{*}$ and Miguel A. Peinado$^{**}$ \\
\\
% \begin{small}
% Institute of Predictive and Personalized Medicine of Cancer (IMPPC) \\
% Health Science Research Institute of the Germans Trias i Pujol Foundation (IGTP)\\
 $^{*}$ Equal contributors \\
 $^{**}$ Corresponding author \texttt{map@imppc.org}
% \end{small}
}

\date{\today}

\begin{document}

\maketitle

% \section{Paper}

\begin{abstract}
The TCGA Wanderer (\url{http://maplab.cat/betawanderer}) is a very intuitive Web tool allowing real time access and visualization of gene expression and DNA methylation profiles from The Cancer Genome Atlas (TCGA) data using gene targeted queries. Given a gene query, the Web resource provides its expression and DNA methylation. Tumor and normal tissue data are easily compared, giving graphical outputs to aid researchers to explore alterations across samples. The tool takes advantage of the Illumina Infinium 450k platform to scrutinize the probes inside or in the vicinity of the queried gene. Gene expression is inspected exon by exon. The simple interface eases cancer genomics profiling to experimentalists and clinicians without bioinformatics expertise.

\end{abstract}


Data analysis on human cancer is a hotspot in basic and clinical research. As a group of largely heterogenous and complex disease, cancer research is benefited by coordinated efforts on data acquisition and analysis. The Cancer Genome Atlas (TCGA) is a result of an international effort that offers a multilayered view of the genomics and epigenomics of more than 30 human cancer types \cite{weinstein2013cancer}. This enormous data availability opens a wide world of possibilities to the research community. For instance, it allows \textit{in silico} hypothesis testing or validation of in-house results through an independent series.\\


The TCGA serves the information in three tiers according to the degree of its bioinformatical processing: raw, lightly and heavily analyzed data (named levels 1, 2 and 3). Data not belonging uniquely to an individual, such as level 3 and de-identified clinical data, is open access; on the other hand, the download of protected data, such as raw sequencing files, requires authentication via the The Cancer Genomics Hub (CGHub)\cite{wilks2014cancer}. Finally, although the open access data tier offers immediate access to the information, some projects restrict the dissemination of the results brought by its use until a first global analysis is published by the consortia.\\

The complex and comprehensive  nature of the generated data represents a privileged opportunity for researcher to get insights into the molecular landscapes of cancers. In fact, the landmark papers reporting the original data of each tumor type may just be considered a modest anticipation of the expected overall outcome. Nevertheless,  wet lab experimentalists have very limited access to these data because sophisticated bioinformatic skills are required to manage and analyze such large datasets. Hence, development of interfaces facilitating the visualization and analysis of the information to non-bioinformaticians is essential to get a broad exploitation of this gigantic effort.\\

The TCGA Data Portal is the gateway to retrieve the TCGA information. Basically, offers a text file for each processed sample belonging to a dataset subjected to a platform; compressed packages for multiple samples are allowed. For instance, it allows downloading 992 files with the clinical data related to samples of the Colon Adenocarcinoma (COAD) dataset, but only provides RNAseq results for 303 of them. It is worth noting that as the number of files and their sizes increase, further processing of this information exceeds the spredsheet-based analysis capabilities and requires some amount of bioinformatics knowledge, (i.e. programming under the R statistical environment). Comprehensive and integrative analysis is indeed dependent on informatics solutions, such as TCGA Assembler \cite{zhu2014tcga} or the cBioPortal\cite{gao2013integrative}.\\


The availability of such a large amount of data is exciting to the cancer research community as it helps to check candidate \textit{loci}. For instance, as some genes show aberrant promoter hypermethylation and therefore silencing during the tumor progression \cite{jones2002fundamental}, experimentalists might query the expression and methylation profiles of their candidate \textit{loci} on tumoral and normal matched tissues. However, the Data Portal does not allow to query individual genomic locations but rather offers bulky, whole genome datasets. To our knowledge, although much effort has been done on facilitating the access to TCGA data, none of the published tools allows an user-friendly interface to solve this kind of queries. We believe that this candidate and exploratory approaches might be particularly benefited by using RNASeq and methylation data. Our response has been the development of Wanderer, a very simple and intuitive web tool allowing real time access and visualization of gene expression and DNA methylation profiles from TCGA data using gene targeted queries.\\

% With this aim we developed a tool, the TCGA Wanderer, that allows to easily retrieve and visualize the regional profiles of groups of samples belonging to a tumor type. 

For any given gene selected by the investigator, the tool provides detailed individual profiles of gene expression, exon by exon, and DNA methylation of all the probes inside or in the vicinity of the gene (Figure 1). Graphs for normal and tumor samples from any of the available TCGA datasets are readily produced. Summarizing plots and tables are also generated, allowing further data analysis or representation using any software the investigator is used to (e.g.: Excel or any other spreadsheet or graphing tool). In addition, statistical analysis is applied to identify differential gene expression or DNA methylation between normal and tumor samples at either the single exon or the DNA methylation probe level, respectively. The tool allows local navigation and zoom in/out within the region of interest, as well as simple graph customization. Resulting graphs and tables may be downloaded and an application programming interface (API) allows data sharing, automatable query of multiple instances and direct linkage from external servers.\\
 
Experimentalists and clinicians might assess biological questions such as which region of a CpG island should be analyzed to find the largest abnormal DNA methylation in tumor samples; and whether this one the same for different tumor types. TCGA Wanderer offers a simple and intuitive plot that summarizes the behaviour for a given candidate gene (Figure 2).\\

The user-friendly web application has been developed using R/shiny and is freely accesable at \url{maplab.cat/betawanderer} (provisional address). As it is hosted in a shiny server, the final user only needs a web browser to run it and requires no installation, being platform-indepedent and runnable on commodity computers, such as low-memory laptops.\\


% TODO afegir lo dels boxplots

\section{Acknowledgements}
This work was supported by grants from the Spanish Ministry of Economy and Knowledge (SAF2011/23638, PTA2011-5655-I) and from Generalitat de Catalunya (2009 SGR1356). The application published here is based upon data generated by the TCGA Research Network: http://cancergenome.nih.gov/. 



\section{Competing Interests}
The authors declare no competing interests.


\bibliographystyle{plain}
\bibliography{wanderer_paper}      % Bibliography file (usually '*.bib' )


\section{Figure legends}

\subsection{Figure 1}
Snapshot of Wanderer webpage. Boxes indicate the different panels for gene name/ID and dataset selection (\textbf{a}), view customization (\textbf{b} and \textbf{c}), output (\textbf{d}) and data download and links (\textbf{e}).

\subsection{Figure 2}
DNA methylation profile of PTGIS CpG island in Colon adenocarcinomas (COAD) and Prostate adenocarcinomas (PRAD). Note that cg1077290 probe (*) is already very methylated in colon normal tissue, offering poor discriminat resolution to detect tumor hypermethylation compared with the rest of CpG island probes (labeled in green). In contrast, this probe (cg1077290) shows the highest level of hypermethylation in Prostate cancers (PRAD) and remains unmethylated in normal, resulting as the most discriminant variable in tis type of cancer.

% \section*{Cover letter}


% Dear Editor,\\

% We would like to submit the manuscript entitled \textit{TCGA Wanderer: integration and visualization of TCGA Data} for consideration as a Correspondence letter to Nature Methods.\\

% The Cancer Genome Atlas (TCGA) iniciative is an exciting oportunity to check comprehensively the genomic and epigenomic abnormalities on clinically characterized cancersamples. However, the easy access to this data is far away from the wet lab experimentalists, as the amount of data offered is very large.\\

% Particularly, analysis of DNA methylation is probably one of the hottest issues in biomedical research: in 2013 about 900 papers included this term in the title and about 4000 in the abstract, with an annual growth rate $\ge$ 20\% during the last 5 years (data from PubMed). The abnormalities on gene expression and its coupling to DNA methylation during tumorogenesis are striking features that have been atracting researchers attention since the 2000.\\

% Here we describe a new Web-tool that provides an easy-to-use interface to fetch and visualize expression and methylation data from the TCGA. The user is only asked to introduce the tumor type (i.e. prostate cancer), whether the query is related to methylation or expression, and the gene symbol or Ensembl ID. Then, the tool retrieves and plots the results in a highly reactive fashion, providing a dynamic image rendering that allows the generation of publication-ready plots. The outputs are easy to download, including both the plots and the spreadsheets.\\

% To our knowledge this is the first tool that offers this capabilities. We are convinced it will be of great interest for a broad spectrum Nature Methods readers.\\


% Sincerely, \\
% Izaskun Mallona, PhD.

\end{document}